% poemscompat: A single set of commands for typesetting poetry that use either
% poetrytex or poemscol.
%
% Copyright 2013-2016 Jonathan Lovelace
%
% This file may be distributed and/or modified under the conditions of the
% LaTeX Project Public License, either version 1.3c of this license or (at your
% option) any later version. The latest version of this license is in:
%
% http://www.latex-project.org/lppl.txt
%
% and version 1.3c or later is part of all ditributions of LaTeX version
% 2008/05/04 or later.
%
% Maintainer: Jonathan Lovelace
% Website:    https://shinecycle.wordpress.com
% Contact:    kingjon3377@gmail.com
%
% This work consists of this file poemscompat.tex; it is distributed with, and
% a dependency of, but distinct from the "poetry ebook utilities"

% hard dependency of poemscol, "for managing the running headers," but we want
% to use it for that even with poetrytex.
\usepackage{fancyhdr}
% Provides \ifempty, used below.
\usepackage{etoolbox}

% The first argument to the \poemscompat command will be run if POEMSCOL is
% defined, and the second if POETRYTEX is defined.
\newcommand{\poemscompat}[2]{%
	\ifdefined\POEMSCOL%
		\ifdefined\POETRYTEX%
			\PackageError{poemscompat}%
				{Two incompatible packages requested}%
				{Either \backslash{}POEMSCOL or \backslash{}POETRYTEX must be defined, but not both}%
		\else%
			#1%
		\fi%
	\else%%
		\ifdefined\POETRYTEX%
			#2%
		\else%
			\PackageError{poemscompat}%
				{Choose which package to use}%
				{Either \backslash{}POEMSCOL or \backslash{}POETRYTEX must be defined}%
		\fi%
	\fi%
}

% Set up the non-common parts of the environment
\poemscompat{%
	% poemscol dependencies:
	% "to simplify page geometry" (with `geometry')
	\usepackage{ifthen,keyval}%
	% "for managing the index"
	\usepackage{imakeidx}
	% The package we use to typeset the collection
	\usepackage{poemscol}%
	\indexingontrue%
	\newcommand*{\ptdedication}{Renew \textsf{\textbackslash{}ptdedication}}
	\newcommand*{\makededication}[1][flushright]{%
		\thispagestyle{empty}%
		\vspace*{\prededicationvspace}%
		\begin{#1}%
			\beforededication{\dedicationformat\ptdedication}\afterdedication%
		\end{#1}%
		\vspace*{\postdedicationvspace}%
	}%
	\newcommand*{\dedicationformat}{\normalfont\itshape}%
	\newlength{\prededicationvspace}%
	\newlength{\postdedicationvspace}%
	\setlength{\prededicationvspace}{\fill}%
	\setlength{\postdedicationvspace}{\fill}%
	\newcommand*{\beforededication}{}%
	\newcommand*{\afterdedication}{}%
	% To allow defining a subtitle
	\usepackage{titling}%
	% From https://tex.stackexchange.com/a/50186 by user "egreg"; this
	% snippet may be shared under CC-BY-SA 3.0
	\newcommand{\subtitle}[1]{%
		\posttitle{%
			\par\end{center}
			\begin{center}\large#1\end{center}
			\vskip0.5em%
		}%
	}%
}{%
	% Defines the symbol we're using to separate titleless poems
	% from the preceding ones
	\usepackage{textcomp}%
	% The package we use to typeset the collection
	\usepackage{poetrytex}%
	% Needed to fix poem-group styling, below
	\usepackage{sectsty}%
}

% The \compattitle command sets up either package's title-page knowledge. The
% first argument is the author, the second the title. If using poetrytex and
% wishing for a subtitle, change \ptsubtitle after calling this.
\newcommand{\compattitle}[2]{%
	\poemscompat{%
		\author{#1}%
		\title{#2}%
	}{%
		\renewcommand*{\pttitle}{#2}%
		\renewcommand*{\ptauthor}{#1}%
		\renewcommand*{\ptsubtitle}{\relax}%
	}%
}

% The \compatfulltitle command sets up either package's title-page knowledge.
% The first argument is the author, the second the title, and the third the
% subtitle.
\newcommand{\compatfulltitle}[3]{%%
	\poemscompat{%
		\author{#1}%
		\title{#2}%
		\subtitle{#3}%
	}{%
		\renewcommand*{\pttitle}{#2}%
		\renewcommand*{\ptauthor}{#1}%
		\renewcommand*{\ptsubtitle}{#3}%
	}%
}

% Boilerplate reduction. The first argument is the title of the poem, the
% second is its subtitle, the third is what the poem should be called if it has
% no title---usually its incipit. Any may be blank.
\let\defpoem\poem{}
\let\enddefpoem\endpoem{}

\poemscompat{%
	\renewenvironment{poem}[3]{%
		\ifblank{#3}{\relax}{\poemtitlenotitle{#3}}%
		\ifblank{#1}{\relax}{\poemtitle{#1}}%
		\ifblank{#2}{\relax}{\poemsubtitle{#2}}%
		\defpoem}{\enddefpoem}%
	\newcommand{\firstline}[1]{#1}%
	\newcommand{\bangfirstline}[2]{#2}%
	\newcommand{\bangincipit}[2]{#2}%
	\newcommand{\incipitindex}[1]{#1}%
	\newcommand{\titleindex}[1]{\relax}%
}{%
	\renewenvironment{poem}[3]{\pagebreak[2]\defpoem{#1}{#2}\nopagebreak\ifblank{#3}{\relax}{\incipit*{#3}}\nopagebreak}{\enddefpoem}%
	\newcommand{\firstline}[1]{\index[titles]{#1}\nopagebreak #1}%
	\newcommand{\bangfirstline}[2]{\index[titles]{#1}\nopagebreak #2}%
	\newcommand{\bangincipit}[2]{\index[titles]{#1}\nopagebreak\incipit{#2}}%
	\newcommand{\incipitindex}[1]{\index[titles]{#1}\nopagebreak\incipit{#1}}%
	\newcommand{\titleindex}[1]{\index[titles]{#1}\nopagebreak}%
}

% Compatibiity with the other package
\poemscompat{%
	\newcommand{\verselinenb}{\verseline}
}{%
	\newcommand{\verseline}{\\}%
	\newcommand{\verselinenb}{\\*}%
}
\makeatletter
\poemscompat{%
	\newcommand{\ptspacer}{\relax}%
	\newcommand{\incipit}[1]{#1}%
	\newcommand{\silentincipit}[1]{\relax}%
	\renewcommand{\sequencetitlepenalty}{\cleardoublepage}
	\renewcommand{\aftersequencetitleskip}{\bigskip}
	\renewcommand{\sequencetitlefont}{\fontsize{20}{24}\selectfont}
}{%
	\newenvironment{stanza}[0]{\relax}{\relax}%
	\sectionfont{\nohang\centering}%
	\newcommand{\sequencetitle}[1]{\phantomsection\poemgroup{#1}}%
	\def\sequencesectiontitle{\@ifstar\@starredsequencesectiontitle\@sequencesectiontitle}%5
	\newcommand{\@starredsequencesectiontitle}[1]{\phantomsection\thispagestyle{empty}\subsection{#1}}%
	\newcommand{\@sequencesectiontitle}[1]{\phantomsection\thispagestyle{empty}\subsection{#1}\addcontentsline{top}{subsection}{#1}}%
	\newcommand{\sequencefirstsectiontitle}[1]{\sequencesectiontitle{#1}}%
	\newcommand{\silentincipit}[1]{\incipit*{#1}}%
}
\makeatother

% A custom inclusion mechanism for poems, to reduce the amount of boilerplate
% The first argument is the file to include. The second is the title to index
% the poem under---usually only needed if the title includes a literal pound
% sign or is italicized. The third is its title. The fourth is its subtitle.
% The fifth is its incipit, or whatever else to call it if it has no title. You
% need to give the first argument and either the third or the fifth.
\newcommand{\includepoem}[5]{%
	\begin{poem}%
	{#3}{{\footnotesize #4}}{#5}%
	\titleindex{\ifblank{#2}{\ifblank{#3}{#5}{#3}}{#2}}%
	\input{#1}%
	\end{poem}%
}

% An alternate form of \includepoem, taking as an additional argument an
% alternate form of the title or incipit to additionally index it under.
\newcommand{\includepoemalternateindex}[6]{%
	\begin{poem}%
	{#3}{#4}{#5}%
	\titleindex{\ifblank{#2}{\ifblank{#3}{#5}{#3}}{#2}}%
	\titleindex{#6}%
	\input{#1}%
	\end{poem}%
}

% A command to make something centered. The name is because it is used for a
% "section subtitle"/"section epigram"/"section explanation", but it could be
% used for other purposes.
\newcommand{\epigram}[1]{\begin{center}#1\end{center}}

% A command to render the title and copyright (back-of-title-page) pages. This
% takes the following arguments:
% #1 the copyright text (such as "Text and design \textcopyright{} YYYY
%    \theauthor", assuming 'titling' for \theauthor)
% #2 the publication date
% #3 the ISBN-13
% #4 the ISBN-10
\newcommand{\titleandcopyright}[4]{%
	\pagenumbering{gobble}%
	\cleardoublepage%
	\pagestyle{empty}%
	\begin{titlepage}%
	\date{}%
	\maketitle%
	\clearpage%
	\null%
	\vfill%
	\noindent #1

	\noindent Published #2

	\noindent ISBN-13: #3 \\%
	\noindent ISBN-10: #4%
	\end{titlepage}%
}
