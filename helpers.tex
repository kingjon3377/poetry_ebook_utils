% The main trick here comes from http://tex.stackexchange.com/a/17813, by
% StackExchange user Aditya; the code in this file is thus used under the
% CC-BY-SA license version 3.0 <http://creativecommons.org/licenses/by-sa/3.0/>
% Some further extension by Jonathan Lovelace.

% Set up a way to make illustrations as big as possible without overflowing the page
\newcommand\measurepage{\dimexpr\pagegoal-\pagetotal-\baselineskip\relax}
\newcommand\illustration[1]{\label{#1}\begin{center}\includegraphics[height=\measurepage,width=\textwidth,keepaspectratio]{#1}\end{center}}

\newcommand\fixedheightillustration[2]{\label{#1}\begin{center}\includegraphics[height=#2,width=\textwidth,keepaspectratio]{#1}\end{center}}

\let\canonicalillustration\illustration
\let\canonicalfhillustration\fixedheightillustration

% \disablegraphics turns our \illustration command into \relax; \enablegraphics reverses this.
\newcommand{\enablegraphics}{\renewcommand{\illustration}[1]{\canonicalillustration}\renewcommand{\fixedheightillustration}[2]{\canonicalfhillustration}}
\newcommand{\disablegraphics}{\renewcommand{\illustration}[1]{\label{##1}\relax}\renewcommand{\fixedheightillustration}[2]{\label{##1}\relax}}

% This trick comes from https://tex.stackexchange.com/a/600, by StackExchange
% user lockstep; it is used under CC-BY-SA license version 2.5.
% This creates a 'changemargin' environment, which takes the right and left
% margin (or maybe inner and outer?) figures as first and second parameters.

\def\changemargin#1#2{\list{}{\rightmargin#2\leftmargin#1}\item[]}
\let\endchangemargin=\endlist
